\documentclass{article}
\usepackage{listings}
\usepackage{color}
\usepackage{listings}
\usepackage{xcolor}
\usepackage{hyperref}


% Definizione dei colori per il codice sorgente
\definecolor{mygray}{rgb}{0.5,0.5,0.5}
\definecolor{mygreen}{rgb}{0,0.6,0}
\definecolor{myorange}{rgb}{1.0,0.4,0}

\lstset{
    language=Python,
    basicstyle=\footnotesize\ttfamily,
    keywordstyle=\color{blue},
    commentstyle=\color{mygray},
    stringstyle=\color{mygreen},
    showstringspaces=false,
    tabsize=2,
    breaklines=true,
    breakatwhitespace=false
}

\begin{document}
\tableofcontents

 
\title{Documentazione per Ryno}
\author{Senni Mattia}
\maketitle

\section{Introduzione}
Ryno è un semplice server HTTP in Python, ispirato al framework Express di Node.js. Questo documento fornisce una guida dettagliata su come utilizzare e comprendere Ryno, descrivendo le sue funzionalità, il funzionamento interno e le considerazioni aggiuntive.

\section{Funzionalità}
Di seguito sono elencate le principali funzionalità di Ryno:

\begin{itemize}
  \item \textbf{Middleware Based}: Ryno è middleware based, prendendo ispirazione dal framework Express di Node.js. Ciò significa che le funzioni di middleware possono essere aggiunte e utilizzate per gestire le richieste HTTP in modo modulare.
  
  \item \textbf{Servizio di file statici}: Ryno è in grado di servire file statici dal dispositivo. Il server può essere configurato per servire una specifica cartella e rendere disponibili i file in essa contenuti su una determinata route.
  
  \item \textbf{Sistema di routing interno}: Ryno ha un sistema di routing interno che consente di definire facilmente le rotte per gestire le richieste HTTP.
  
  \item \textbf{Gestione degli header e dello status della risposta}: Ryno gestisce correttamente gli header e lo status della risposta HTTP, consentendo un controllo completo sulla risposta inviata al client.
  
  \item \textbf{Supporto per la compressione gzip}: Ryno supporta la compressione gzip se supportata dal client, riducendo la dimensione dei file inviati e migliorando le prestazioni della rete.
  
  \item \textbf{Esempio di sito funzionante}: Ryno include un complesso sito web funzionante come esempio, che comprende immagini, font, video, HTML, CSS e JavaScript.
  
  \item \textbf{Supporto per le API}: Ryno supporta la creazione di API.
\end{itemize}

\section{Funzionamento}
Ryno è progettato per essere semplice da utilizzare e configurare. Di seguito sono riportati i passaggi principali per utilizzare Ryno:

\begin{enumerate}
  \item \textbf{Installazione}: Per utilizzare Ryno, è necessario installare Python 3 sul proprio sistema. Dopodiché, è possibile scaricare e utilizzare il codice sorgente di Ryno dalla seguente repository \href{https://github.com/mtttia/Ryno}{Ryno python server su github}.
  
  \item \textbf{Configurazione}: Ryno può essere configurato modificando il file di configurazione presente in src/kernel/configuration.py, specificando la cartella contenente i file statici da servire e definendo le rotte per gestire le richieste HTTP.
  
  \item \textbf{Avvio del server}: Una volta configurato, è possibile avviare il server eseguendo il file server.py tramite python (`python server.py'). Il server si metterà in ascolto delle richieste HTTP in arrivo e risponderà di conseguenza.
  
  \item \textbf{Gestione delle richieste}: Ryno gestirà automaticamente le richieste HTTP in arrivo, instradandole alle funzioni di middleware appropriate e restituendo le risposte corrispondenti.
  
  \item \textbf{Middleware configurabili}: Tramite il file `src/main' è possibile aggiungere nuovi middleware su nuove route per o servire nuovi file statici o creare nuove API, seguendo gli esempi disponibili.
\end{enumerate}
\subsection{Requisiti}
\begin{enumerate}
    \item Python 3 installato
\end{enumerate}

\section{Funzionamento interno}
Analisi struttura delle cartelle e dei file di Ryno:
\begin{enumerate}
    \item server.py: il file principale che avvia il server.
    \item src/main.py: il file che contiene le funzioni di middleware e le rotte per gestire le richieste HTTP.
    \item src/kernel/conf.py: il file di configurazione che definisce l'host e la porta della socket.
    \item src/kernel/errors.py: funzioni utili da richiamare per restituire gli errori HTTP (in particolare sono gestiti il codice 404 e 500).
    \item src/kernel/handle-request.py: prende le richieste in arrivo, scompatta l'oggetto request, carica i middleware, esegue i middleware della route specificata, incapsula seguendo il protocollo http la risposta, la invia al client e chiude la socket.
    \item src/kernel/kernel.py: contiene il codice per aprire la socket e gestire tramite Thread le richieste in entrata.
    \item src/kernel/middleware.py: contiene funzioni che facilitano la creazione di un middleware generico + la funzione per aggiungere un middleware che consente di gestire le risposte tramite risorse statiche e la relativa compressione tramite gzip se presente l'header apposito nella richiesta.
    \item src/kernel/response.py: contiene dei metodi che facilitano il set dei parametri della risposta http (tramite un pattern che ricorda il Builder pattern).
    \item static/**: contiene i file statici che il server può servire (configurato con una complessa landing page per dimostrare il funzionamento del server), data la dimensione dei suoi file è consigliata la visualizzazione del sito tramite un browser che supporti la compressione gzip.
\end{enumerate}

\subsection{Esempio di middleware}
Creazione di un middleware per gestire le risposte custom (nell file src/main.py):
\begin{lstlisting}
    
def main(request, response):
    addMiddleware(response, "api", testMiddlewareBody)

def testMiddlewareBody(response, server):
    set_body(response, "{\"ping\":\"pong\"}")
    add_header(response, "Content-Type", "application/json")
    return True
    
\end{lstlisting}
Analisi di request e response:
Gli oggetti request e response contengono informazioni riguardo la richiesta e la risposta da dare.
Il return True o False decreta se quello deve essere l'ultimo middleware da visualizzare oppure no
\begin{lstlisting}
    
    response = {
        "body": "",
        "body_is_in_bytes": False,
        "status": 200,
        "status_message": "OK",
        "protocol": "HTTP/1.1",
        "headers": [], # contiene tutti gli header
        "middleware": [] # contiene tutti i middleware ordinati  
    }

    request = {
        "path": "index.html", # il file richiesto
        "folder": "", # la cartella richiesta
        "method": "GET", # il metodo usato per la richiesta
        "get": {"page":1}, # i parametri get
        "headers": {}, # tutti gli header della richiesta
    }
\end{lstlisting}
Attualmente Ryno non gestisce la lettura del body della richiesta
\section{Considerazioni aggiuntive}
Ryno è progettato per essere flessibile e personalizzabile, consentendo agli sviluppatori di creare facilmente servizi web robusti e scalabili (partendo da un sito statico a delle funzionanti API).

\end{document}
